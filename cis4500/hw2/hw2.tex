\documentclass[12pt,letterpaper, onecolumn]{exam}
\usepackage{graphicx}
\usepackage{caption}
\usepackage{listings}
\usepackage[dvipsnames]{xcolor}
\graphicspath{ {./images/} }
\lhead{Mustafa Rashid\\}
\rhead{Homework 2: Relational DB Design\\}
\chead{\hline} 
\thispagestyle{empty} 
\newcommand*{\setdef}[1]{\left\{#1 \right\}} 
\renewcommand{\thepartno}{\Alph{partno}} % Force uppercase letters
\renewcommand{\partlabel}{\thepartno.}    % Format as A. B. C.

\begin{document}
\begingroup  
\noindent\LARGE CIS 5500: Database and Information Systems\\
\noindent\LARGE Homework 2: Relational DB Design\\
\noindent\large \today\\
\noindent\large Mustafa Rashid\par
\noindent\large Spring 2026\par
\endgroup
\rule{\textwidth}{0.4pt}
\pointsdroppedatright
\printanswers
\renewcommand{\solutiontitle}{\noindent\textbf{Ans:}\enspace}

% SQL listing from https://gist.githubusercontent.com/vivngo/e37e1c7b79bf7e8b910c6566c59c46be/raw/db8723fc80c8b1b01665851106dba2f370b5d865/code.tex
\usepackage{color}
\definecolor{dkgreen}{rgb}{0,0.6,0}
\definecolor{gray}{rgb}{0.5,0.5,0.5}
\definecolor{mauve}{rgb}{0.58,0,0.82}
\lstset{language=SQL,
  basicstyle={\small\ttfamily},
  belowskip=3mm,
  breakatwhitespace=true,
  breaklines=true,
  classoffset=0,
  columns=flexible,
  commentstyle=\color{dkgreen},
  framexleftmargin=0.25em,
  frameshape={}{}{}{}, %To remove to vertical lines on left, set `frameshape={}{}{}{}`
  keywordstyle=\color{blue},
  numbers=none, %If you want line numbers, set `numbers=left`
  numberstyle=\tiny\color{gray},
  showstringspaces=false,
  stringstyle=\color{mauve},
  tabsize=3,
  xleftmargin =1em
}

\begin{questions}
  \question \textbf{Question 1 (30 points)}
  \begin{parts}
    \part \textbf{(20 points)}
     \begin{center}
       \includegraphics[scale=0.25]{hw2_q1.png}
     \end{center}
    \part \textbf{(5 points)}
    \begin{solution}
      It is possible to have a start or end stop that does not have a TransportLine assigned to it. We can add a not exists constraint to the stations
      table to check that a station can be a stop if and only if it exists in the TransportLine table \textcolor{red}{double check this!}
    \end{solution}
    \part \textbf{(5 points)}
    \begin{solution}
      Add an attribute to the relationship \texttt{serves} that is called \texttt{no\_of\_platforms}. We also add an attribute to the Station entity set 
      \texttt{maximum\_no\_of\_platforms}. We add a constraint that ensures that \texttt{no\_of\_platforms} is less than or equal to \texttt{maximum\_no\_of\_platforms}.
    \end{solution}
  \end{parts}

  \pagebreak

  \question \textbf{Question 2 (20 points)}\begin{parts}
    \part \textbf{12 points}
    \begin{solution}\\
      \begin{lstlisting}
        CREATE TABLE Artists(
        ArtistID INT PRIMARY KEY,
        Name VARCHAR(50) NOT NULL,
        Nationality VARCHAR(50) NOT NULL,
        BirthYear INT(4) NOT NULL
        );

        CREATE TABLE Customers(
        CustomerID INT PRIMARY KEY,
        Name VARCHAR(50) NOT NULL,
        Email VARCHAR(100) NOT NULL
        );

        CREATE TABLE Artworks(
        ArtworkID INT PRIMARY KEY,
        OwnerID INT FOREIGN KEY REFERENCES Customers(CustomerID),
        CreatorID INT FOREIGN KEY REFERENCES Artists(ArtistID),
        Title VARCHAR(50) NOT NULL,
        AYear INT(4),
        Medium VARCHAR(50)
        );

        CREATE TABLE Exhibitions(
        ExhibitionID INT PRIMARY KEY,
        ExhibitionName VARCHAR(100) NOT NULL
        );

        CREATE TABLE DisplayedIn(
        ArtworkID INT,
        ExhibitionID INT,
        PRIMARY KEY (ArtworkID, ExhibitionID),
        FOREIGN KEY (ArtworkID) REFERENCES Artworks(ArtworkID),
        FOREIGN KEY (ExhibitionID) REFERENCES Exhibitions(ExhibitionID),
        StartDate DATE,
        EndDate DATE
        );
      \end{lstlisting}

    \end{solution}
    \part \textbf{4 points}
    \part \textbf{4 points}
  \end{parts}
  \question \textbf{Question 3 (40 points)}\begin{parts}
    \part \textbf{4 points}
    \part \textbf{4 points}
    \part \textbf{6 points}
    \part \textbf{5 points}
    \part \textbf{2 points}
    \part \textbf{2 points}
    \part \textbf{12 points}
    \part \textbf{5 points}
  \end{parts}
\end{questions}
\end{document}
