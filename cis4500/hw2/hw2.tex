\documentclass[12pt,letterpaper, onecolumn]{exam}
\usepackage{graphicx}
\usepackage{caption}
\usepackage{color}
\usepackage{listings}
\usepackage[dvipsnames]{xcolor}
\graphicspath{ {./images/} }
\lhead{Mustafa Rashid\\}
\rhead{Homework 2: Relational DB Design\\}
\chead{\hline} 
\thispagestyle{empty} 
\newcommand*{\setdef}[1]{\left\{#1 \right\}} 
\renewcommand{\thepartno}{\Alph{partno}} % Force uppercase letters
\renewcommand{\partlabel}{\thepartno.}    % Format as A. B. C.

\begin{document}
\begingroup  
\noindent\LARGE CIS 5500: Database and Information Systems\\
\noindent\LARGE Homework 2: Relational DB Design\\
\noindent\large \today\\
\noindent\large Mustafa Rashid\par
\noindent\large Spring 2026\par
\endgroup
\rule{\textwidth}{0.4pt}
\pointsdroppedatright
\printanswers
\renewcommand{\solutiontitle}{\noindent\textbf{Ans:}\enspace}

% SQL listing from https://gist.githubusercontent.com/vivngo/e37e1c7b79bf7e8b910c6566c59c46be/raw/db8723fc80c8b1b01665851106dba2f370b5d865/code.tex
\definecolor{dkgreen}{rgb}{0,0.6,0}
\definecolor{gray}{rgb}{0.5,0.5,0.5}
\definecolor{mauve}{rgb}{0.58,0,0.82}
\lstset{language=SQL,
  basicstyle={\small\ttfamily},
  belowskip=3mm,
  breakatwhitespace=true,
  breaklines=true,
  classoffset=0,
  columns=flexible,
  commentstyle=\color{dkgreen},
  framexleftmargin=0.25em,
  frameshape={}{}{}{}, %To remove to vertical lines on left, set `frameshape={}{}{}{}`
  keywordstyle=\color{blue},
  numbers=none, %If you want line numbers, set `numbers=left`
  numberstyle=\tiny\color{gray},
  showstringspaces=false,
  stringstyle=\color{mauve},
  tabsize=3,
  xleftmargin =1em
}

\begin{questions}
  \question \textbf{Question 1 (30 points)}
  \begin{parts}
    \part \textbf{(20 points)}
     \begin{center}
       \includegraphics[scale=0.25]{hw2_q1.png}
     \end{center}
    \part \textbf{(5 points)}
\begin{solution}
A TravelSegment could start or end at a Station that is not served by its associated TransportLine. 
This would be valid in the ER diagram because there is no constraint requiring the start and end 
stations of a TravelSegment to be among the stations served by that TransportLine.

To enforce this constraint, we must require that the start and end stations of a TravelSegment 
appear in the Serves relationship with the same TransportLine. In other words, a TravelSegment 
may only start and end at stations that are served by its associated TransportLine.
\end{solution}
    \part \textbf{(5 points)}
\begin{solution}
Add an attribute to the relationship \texttt{Serves} called \texttt{no\_of\_platforms}. 
Add an attribute to the Station entity set called \texttt{maximum\_no\_of\_platforms}. 

We then add a constraint that, for each Station, the sum of \texttt{no\_of\_platforms} 
across all TransportLines assigned to that Station does not exceed\\
\texttt{maximum\_no\_of\_platforms}.
\end{solution}
  \end{parts}

  \pagebreak

  \question \textbf{Question 2 (20 points)}\begin{parts}
    \part \textbf{12 points}
    \begin{solution}\\
      \begin{lstlisting}
        CREATE TABLE Artists(
        ArtistID INT PRIMARY KEY,
        Name VARCHAR(50) NOT NULL,
        Nationality VARCHAR(50) NOT NULL,
        BirthYear INT(4) NOT NULL
        );

        CREATE TABLE Customers(
        CustomerID INT PRIMARY KEY,
        Name VARCHAR(50) NOT NULL,
        Email VARCHAR(100) NOT NULL
        );

        CREATE TABLE Artworks(
        ArtworkID INT PRIMARY KEY,
        OwnerID INT FOREIGN KEY REFERENCES Customers(CustomerID),
        CreatorID INT FOREIGN KEY REFERENCES Artists(ArtistID),
        Title VARCHAR(50) NOT NULL,
        AYear INT(4),
        Medium VARCHAR(50)
        );

        CREATE TABLE Exhibitions(
        ExhibitionID INT PRIMARY KEY,
        ExhibitionName VARCHAR(100) NOT NULL
        );

        CREATE TABLE DisplayedIn(
        ArtworkID INT,
        ExhibitionID INT,
        PRIMARY KEY (ArtworkID, ExhibitionID),
        FOREIGN KEY (ArtworkID) REFERENCES Artworks(ArtworkID),
        FOREIGN KEY (ExhibitionID) REFERENCES Exhibitions(ExhibitionID),
        StartDate DATE,
        EndDate DATE
        );
      \end{lstlisting}
    \end{solution}
    \pagebreak
    \part \textbf{4 points}
    \begin{solution}
      We can create an assertion as follows
      \begin{lstlisting}
        CREATE ASSERTION exhibition_cardinality
CHECK (
    NOT EXISTS (
        SELECT ExhibitionID
        FROM DisplayedIn
        GROUP BY ExhibitionID
        HAVING COUNT(*) NOT BETWEEN 5 AND 20
    )
);
\end{lstlisting}
    \end{solution}
    \part \textbf{4 points}
    \begin{solution}
      We change the cardinality on the Artists entity set from \(0..*\) to \(1..*\). We also add the following assertion
      \begin{lstlisting}
CREATE ASSERTION artist_has_artwork
CHECK (
  NOT EXISTS (
    SELECT *
    FROM Artists A
    WHERE NOT EXISTS (
      SELECT *
      FROM Artworks W
      WHERE W.CreatorID = A.ArtistID
    )
  )
);
\end{lstlisting}
    \end{solution}
  \end{parts}
  \pagebreak
  \question \textbf{Question 3 (40 points)}\begin{parts}
    \part \textbf{4 points}
    \begin{solution}
      No they cannot.\\If they did then the functional dependencies PassengerID \(\rightarrow\) PassengerEmail and PassengerEmail \(\rightarrow\) PassengerID would not hold.
    \end{solution}
    \part \textbf{4 points}
    \begin{solution}
Yes. The functional dependency TrainNo \(\rightarrow\) ConductorID only ensures each train has one conductor, but it does not prevent multiple trains from having the same conductor.
\end{solution}
    \part \textbf{6 points}
    \begin{solution}
      The candidate keys are \{TrainNo, PassengerID\} and \{TrainNo, PassengerEmail\}. 
TrainNo determines DepartureDate and ConductorID, PassengerID determines PassengerEmail, and together TrainNo and PassengerID determine SeatNo and BaggageQty, giving all attributes. Since PassengerEmail \(\leftrightarrow\) PassengerID, replacing PassengerID with PassengerEmail also forms a key.
\end{solution}
    \part \textbf{5 points}
    \begin{solution}
The dependency TrainNo \(\rightarrow\) DepartureDate, ConductorID violates 3NF because TrainNo is not a superkey 
and neither DepartureDate nor ConductorID is a member of any candidate key. 
All other dependencies satisfy 3NF because either the left-hand side is a superkey or 
the right-hand side attribute is a member of some candidate key.
\end{solution}
    \part \textbf{2 points}
    \begin{solution}
      \textcolor{red}{FINISH ME!}
    \end{solution}
    \part \textbf{2 points}
    \begin{solution}
      \textcolor{red}{FINISH ME!}
    \end{solution}
    \part \textbf{12 points}
    \begin{solution}
      \textcolor{red}{FINISH ME!}
    \end{solution}
    \part \textbf{5 points}
    \begin{solution}
      \textcolor{red}{FINISH ME!}
    \end{solution}
  \end{parts}
\end{questions}
\end{document}
