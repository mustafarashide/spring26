\documentclass[12pt,letterpaper, onecolumn]{exam}
\usepackage{graphicx}
\usepackage{caption}
\usepackage[dvipsnames]{xcolor}
\graphicspath{ {./images/} }
\lhead{Mustafa Rashid\\}
\rhead{Homework 2: Relational DB Design\\}
\chead{\hline} 
\thispagestyle{empty} 
\newcommand*{\setdef}[1]{\left\{#1 \right\}} 
\renewcommand{\thepartno}{\Alph{partno}} % Force uppercase letters
\renewcommand{\partlabel}{\thepartno.}    % Format as A. B. C.

\begin{document}
\begingroup  
\noindent\LARGE CIS 5500: Database and Information Systems\\
\noindent\LARGE Homework 2: Relational DB Design\\
\noindent\large \today\\
\noindent\large Mustafa Rashid\par
\noindent\large Spring 2026\par
\endgroup
\rule{\textwidth}{0.4pt}
\pointsdroppedatright
\printanswers
\renewcommand{\solutiontitle}{\noindent\textbf{Ans:}\enspace}  

\begin{questions}
  \question \textbf{Question 1 (30 points)}
  \begin{parts}
    \part \textbf{(20 points)}
     \begin{center}
       \includegraphics[scale=0.25]{hw2_q1.png}
     \end{center}
    \part \textbf{(5 points)}
    \begin{solution}
      It is possible to have a start or end stop that does not have a TransportLine assigned to it. We can add a not exists constraint to the stations
      table to check that a station can be a stop if and only if it exists in the TransportLine table \textcolor{red}{double check this!}
    \end{solution}
    \part \textbf{(5 points)}
    \begin{solution}
      Add an attribute to the relationship \texttt{serves} that is called \texttt{no\_of\_platforms}. We also add an attribute to the Station entity set 
      \texttt{maximum\_no\_of\_platforms}. We add a constraint that ensures that \texttt{no\_of\_platforms} is less than or equal to \texttt{maximum\_no\_of\_platforms}.
    \end{solution}
  \end{parts}
  \question \textbf{Question 2 (20 points)}\begin{parts}
    \part \textbf{12 points}
    \part \textbf{4 points}
    \part \textbf{4 points}
  \end{parts}
  \question \textbf{Question 3 (40 points)}\begin{parts}
    \part \textbf{4 points}
    \part \textbf{4 points}
    \part \textbf{6 points}
    \part \textbf{5 points}
    \part \textbf{2 points}
    \part \textbf{2 points}
    \part \textbf{12 points}
    \part \textbf{5 points}
  \end{parts}
\end{questions}
\end{document}
